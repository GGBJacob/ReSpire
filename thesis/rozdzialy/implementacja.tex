\chapter{Implementacja aplikacji ReSpire}

Projekt: Aplikacja została zaprojektowana z myślą o prostocie i intuicyjności użytkowania. 
Projekt w Figmie

\section{Technologie}
Technologie:
Aplikacja została napisana w jęzku Dart we frameworku Flutter. 
Lorem ipsum dolor sit amet, consectetur adipiscing elit. Vivamus elementum arcu nec blandit aliquam. Integer eros dolor, molestie eget dictum quis, luctus sit amet sapien. Proin dignissim felis in ornare volutpat. Morbi vulputate rutrum efficitur. Ut vehicula vehicula metus, et iaculis tortor mattis vel. Nam blandit, arcu quis ultricies blandit, libero ante commodo augue, in accumsan dui leo at orci. Phasellus in augue et velit pulvinar malesuada ut et sem. Nulla vehicula nibh eu odio sollicitudin sagittis. Praesent condimentum semper neque, tincidunt luctus nisl scelerisque sed. Orci varius natoque penatibus et magnis dis parturient montes, nascetur ridiculus mus.

\section{Schemat plików/klas/modułów \Hania\ \Ola\ \Jakub\ \Karol}

\section{Podział prac w zespole \Jakub}
\section{Edycja treningu \Ola}
\subsection{Menu - trening}
\subsection{Menu - dźwięki}
\subsection{Menu - inne}

\section{Przebieg treningu \Hania}
\subsection{TrainingParser}
Klasa TrainingParser powstała w celu przekształcenia hierarchicznych danych treningowych pobranych z lokalnej bazy danych Hive w ciąg występujących po sobie faz, oparty na skonfigurowanym uprzednio przez użytkownika wzorcu oddychania. Jej zadaniem jest zwracanie kolejnych faz do obiektu TrainingController. Dzięki temu logika przełączania faz i powtórzeń jest odseparowana od interfejsu. 
Przedstawiono poniżej w kodzie \ref{code/parser}

\begin{lstlisting}[caption={main.dart - aplikacja Flutter}, label={code/parser}]
import 'package:flutter/material.dart';
\end{lstlisting}

\subsection{Training Controller}
\subsection{Kółko}
\subsection{Instrukcja}

\section{Dźwięki \Jakub}
\subsection{SoundManager}
\subsection{textToSpeach}

\section{Języki}

\section{Baza danych}

\section{Import i export \Karol}







