\chapter{Implementacja aplikacji ReSpire}

Projekt: Aplikacja została zaprojektowana z myślą o prostocie i intuicyjności użytkowania. 
Projekt w Figmie

\section{Technologie}
Aplikacja została napisana w jęzku Dart we frameworku Flutter.

\section{Schemat plików/klas/modułów \Hania\ \Ola\ \Jakub\ \Karol}

\section{Ciekawe fragmenty kodu \Hania\ \Ola\ \Jakub\ \Karol}

\section{Podział prac w zespole \Jakub}

Prace nad projektem realizowane były przy pełnym zaangażowaniu całego zespołu, którego priorytetem było osiągnięcie wysokiej jakości produktu końcowego. Członkowie grupy dynamicznie reagowali na informacje zwrotne, sprawnie wdrażając sugestie opiekuna oraz nowe koncepcje funkcjonalne. Dzięki bieżącej eliminacji błędów i iteracyjnemu wprowadzaniu udoskonaleń, zapewniono ciągłość procesu wytwórczego oraz stabilny rozwój systemu.
\newline
\subsection{Hanna}
Hanna odpowiadała za projekt oraz implementację kluczowych komponentów służących do optymalizacji procesu pobierania i odtwarzania sesji treningowych - \texttt{TrainingController} oraz \texttt{TrainingParser}. Opracowała również wstępny projekt interfejsu użytkownika w środowisku Figma, który stanowił fundament dla dalszych prac deweloperskich. W warstwie widoku zaimplementowała moduł wizualizacji treningów wraz z walidacją danych wejściowych, a także stworzyła widok szczegółów treningu, stanowiący element nawigacyjny między stroną główną a odtwarzaczem.

\subsection{Aleksandra}
Do zadań Aleksandry należało opracowanie kompletnej identyfikacji wizualnej systemu - zaprojektowanie logotypu oraz zdefiniowanie spójnego motywu graficznego aplikacji (dobór kolorystyki i typografii), który stał się obowiązującym standardem dla wszystkich modułów aplikacji. W warstwie implementacyjnej przygotowała widok strony głównej, a także wzbogaciła interfejs użytkownika o animacje, zwiększające dynamikę i interaktywność aplikacji.

\subsection{Jakub}
Zadania Jakuba związane były z zaprojektowaniem i implementacją całej warstwy obsługi dźwięku w aplikacji. Jego głównym zadaniem było stworzenie architektury silnika audio, opartej na komponentach \texttt{SoundManager} oraz \texttt{SingleSoundManager}. Aby zapewnić płynność działania interfejsu i uniknąć blokowania głównego wątku aplikacji, zaimplementował on asynchroniczną obsługę odtwarzania multimediów. Ponadto zaprojektował kluczowe struktury danych oraz zaimplementował klasę \texttt{PlaylistManager}, zarządzającą logiką list odtwarzania. Dodatkowo wdrożył mechanizm internacjonalizacji (obsługę wielu języków) oraz funkcjonalność pozwalającą użytkownikom na import własnych plików dźwiękowych.

\subsection{Karol}

Karol był odpowiedzialny za implementację mechanizmu modyfikacji treningów w sposób intuicyjny dla użytkowników, a także za funkcjonalność importu i eksportu treningów. W obszarze przetwarzania sygnałów opracował moduł generujący dudnienia różnicowe (binaural beats). Ponadto zajął się optymalizacją wydajności poprzez wdrożenie mechanizmu ładowania zasobów z~wyprzedzeniem przed uruchomieniem treningu. W warstwie interfejsu stworzył edytor list odtwarzania plików audio.

\section{Edycja treningu \Ola}
\subsection{Menu - trening}
\subsection{Menu - dźwięki}
\subsection{Menu - inne}

\section{Przebieg treningu \Hania}
\subsection{Parser}
\subsection{Training Controller}
\subsection{Kółko}
\subsection{Instrukcja}

\section{Dźwięki \Jakub}
\subsection{SoundManager}
\subsection{textToSpeach}

\section{Języki}

\section{Baza danych}

\section{Import i export \Karol}







