\chapter{Implementacja aplikacji mobilnej ReSpire}
Projekt został wykonany w środowisku Flutter za pomocą zestawu narzędzi dla programistów Flutter w wersji 3.27.2 wspierającego język Dart w wersji 3.6.1. Ten zestaw narzędzi dla programistów umożliwia tworzenie aplikacji na platformy Android, iOS, Linux, macOS oraz Windows z wykorzystaniem jednej bazy kodu źródłowego. Wybór Fluttera był podyktowany jego wydajnością, bogatym zestawem narzędzi do tworzenia interfejsów użytkownika oraz dużą dostępnością wsparcia. Zapewnie trwałości danych między uruchomieniami zostało zrealizowane przy użyciu lokalnej, nierelacyjnej bazy danych Hive w wersji 2.2.3, przygotowanej pod środowisko Flutter oraz język Dart.

\section{Technologie}
Aplikacja została napisana w jęzku Dart we frameworku Flutter.

\section{Schemat plików/klas/modułów \Hania\ \Ola\ \Jakub\ \Karol}
Aplikacja ReSpire charakteryzuje się modułową budową, na którą składają się łącznie 63 pliki źródłowe w języku Dart. Projekt obejmuje 80 klas oraz ponad 400 metod realizujących logikę biznesową i interfejs użytkownika. Całkowity rozmiar projektu wynosi nieco ponad 10 kLOC (ang. Kilo Lines of Code – tysięcy linii kodu), z czego blisko 8900 stanowi tzw. czysty kod, określany jako SLOC (ang. Source Lines of Code). Ze względu na taką złożoność struktury, diagramy klas i architektury zostały w niniejszej dokumentacji podzielone na logiczne fragmenty, aby zachować ich czytelność.

\subsection{Klasy treningu} \label{subsec:training_classes}

Na trening składa się 12 klas, z czego główną z nich jest \texttt{Training}. Zawiera ona główne informacje o ćwiczeniu takie jak tytuł, czy opis, a także listę etapów, dźwięki i ustawienia. Struktura dźwięków została opisana w rodziale \ref{sec:Sounds}. 

\begin{figure}[H]
  \centering
  \includegraphics[width=\textwidth]{diagramy/class_diagram.png}
  \caption{Diagram klas treningu}
\end{figure}


\section{Edytor treningów \Karol}

Edytor treningów został wykonany jako dedykowany widok \texttt{TrainingEditorPage}, który operuje bezpośrednio na przekazanym mu przez referencję konkretnym obiekcie \texttt{Training}. Struktura treningu przechowywana jest w odpowiednich obiektach klas (\texttt{TrainingStage}, \texttt{BreathingPhase}, \texttt{Sounds}, \texttt{Settings}), dzięki czemu każda modyfikacja wprowadzona w interfejsie natychmiast zostaje zapisana w pamięci a następnie po przejściu do poprzedniego etapu zostaje przekazana do zapisu stałego w bazie danych. 

Obsługa edytora została podzielona na trzy logiczne panele. Aktualnie wyświetlaną zakładkę można wybrać poprzez komponent \texttt{CustomSlidingSegmentedControl} inspirowany obecnym w komponencie \texttt{segmented} z systemu Apple iOS. Pozwoliło to oddzielić edycję przebiegu treningu, konfigurację dźwięków oraz ustawienia uzupełniające, utrzymując jednocześnie wspólną nawigację i kontrolę zapisów. 

\subsection{Menu - trening}
Pierwsza zakładka \texttt{Trening} odpowiada za logiczny układ hierarchicznej sesji oddechowej. Reprezentowana jest przez listę o zmiennym porządku - \texttt{ReorderableListView}, w której wyświetlane są kolejne etapy treningu (\texttt{TrainingStageTile}). Każdy kafelek etapu treningu umożliwia zmianę jego nazwy, liczby powtórzeń i iteracyjnego przyrostu czasów, a także usunięcie całego etapu treningu za potwierdzeniem w wyskakującym oknie dialogowym zaimplementowanym za pomocą \texttt{AlertDialog}. Wewnątrz każdego kafelka etapu treningu umieszczona jest lista faz oddechu reprezentowana przez sekwencję komponentów \texttt{BreathingPhaseTile} zagnieżdżonych w kolejnej \texttt{ReorderableListView}. Dzięki temu zarówno etapy treningu w obrębie całego treningu jak i fazy oddechu w obrębie etapu można uporządkować w kolejności zgodnej z upodobaniem użytkownika w trybie „przeciągnij za kontrolkę i upuść”. Pola numeryczne panelu edycji etapu treningu (liczba powtórzeń, przyrost) oraz fazy oddechu (czas trwania) wykorzystują \texttt{TextEditingController} oraz \texttt{FocusNode}, aby zatwierdzać zmienione wartości dopiero po utracie skupienia na danym elemencie wprowadzania danych, a dodatkowe przyciski +/- umożliwiają inkrementalne korekty w krokach od 0{,}1 do 0{,}5 s. Dodawanie nowych faz i etapów wywołuje przewinięcie listy oraz zdejmuje skupienie z aktywnych pól, co zostało wprowadzone w celu zapobiegania konfliktom z klawiaturą ekranową. Przy próbie opuszczenia ekranu edycji treningu komponent opakowujący \texttt{WillPopScope} sprawdza, czy wszystkie etapy treningu zawierają przynajmniej jedną fazę, i w razie potrzeby wyświetla komunikat z możliwością wyboru automatycznego usunięcia pustych etapów treningu lub powrotu do edycji treningu oddechowego.

\subsection{Menu - dźwięki}
Druga zakładka służy do zarządzania warstwą dźwiękową. Korzysta w tym celu ze współdzielonego obiektu klasy \texttt{Sounds}. Sprowadza się ona do przypisywania plików dźwiękowych do konkretnych sytuacji w treningu. Zakładka ta dzieli się na 3 grupy. \\

W \texttt{Dźwiękach treningu} użytkownik może ustawić sygnał odliczania, sposób odtwarzania komunikatów o nadejściu kolejnej fazy (brak, per faza, globalnie - wybierane z listy rozwijalnej) i wybrać wskazówki audio zmiany etapu oraz cyklu. \\

W grupie \texttt{Muzyka treningu} może zdefiniować tło muzyczne w jednym z pięciu zakresów wybieranych z listy rozwijalnej: brak, globalnie, per etap, per faza, per faza w danym etapie. W trybie globalnym oraz trybie per etap wykorzystywane są komponenty edycji list odtwarzania. W zakresie globalnym jest to \texttt{PlaylistEditor}, który obsługują listy odtwarzania poprzez przeciąganie pozycji, usuwanie oraz dodawanie nowych ścieżek z okna \texttt{AudioSelectionPopup}. Dla wariantu per etap zastosowano \texttt{StagePlaylistsEditor} (korzystający z \texttt{PlaylistEditor}), przyporządkowujący listy odtwarzania na identyfikatory etapów (\texttt{TrainingStage.id}). We wszystkich miejscach obsługiwanych przez \texttt{AudioSelectionPopup} użytkownik może także zaimportować własne pliki audio poprzez \texttt{file\_picker}; pliki są zapisywane w lokalnym systemie plików aplikacji i natychmiast dostępne. Ponadto wszystkie dźwięki w \texttt{AudioSelectionPopup} można odsłuchiwać dzięki \texttt{SingleSoundManager}. Dodatkowo przewidziano osobne pola wyboru dla muzyki odtwarzanej podczas specjalnych faz przygotowawczej i końcowej. 

W trzeciej grupie zatytułowanej \texttt{Dźwięki Binauralne} aplikacja umożliwia wyłączenie muzyki w tle i zamiast niej włączenie odtwarzania tzw. dudnień binauralnych z regulacją częstotliwościami składowymi tychże dźwięków.

Istotnym aspektem edytora jest również zaawansowana konfiguracja listy odtwarzania dla każdego etapu. Widget \texttt{PlaylistEditor} (Listing \ref{code/editor/playlist}) umożliwia użytkownikom nie tylko wybór plików dźwiękowych, ale także ich intuicyjne porządkowanie. Wykorzystano tu komponent \texttt{ReorderableListView}, który obsługuje gesty przeciągania elementów. Dodatkowo, w celu poprawy doświadczenia użytkowników, czasy trwania utworów są ładowane asynchronicznie, co zapobiega blokowaniu interfejsu podczas odczytu metadanych plików audio.

\begin{figure}[H]
\centering
\begin{lstlisting}[caption={Logika zarządzania listą odtwarzania w uproszczonym kodzie edytora (PlaylistEditor)}, label={code/editor/playlist}]
class _PlaylistEditorState extends State<PlaylistEditor> {
  final Map<String, Duration?> _durationCache = {};

  Future<void> _loadDurations() async {
    for (var sound in widget.playlist) {
      if (!_durationCache.containsKey(sound.name)) {
        final duration = await _soundManager.getSoundDuration(sound.name);
        if (mounted) {
          setState(() {
            _durationCache[sound.name] = duration;
          });
        }
      }
    }
  }

  void _reorderSounds(int oldIndex, int newIndex) {
    if (oldIndex < newIndex) {
      newIndex -= 1;
    }
    final newPlaylist = List<SoundAsset>.from(widget.playlist);
    final item = newPlaylist.removeAt(oldIndex);
    newPlaylist.insert(newIndex, item);
    widget.onChanged(newPlaylist);
  }

  @override
  Widget build(BuildContext context) {
    return Column(
      children: [
        // ... 
        ReorderableListView.builder(
          shrinkWrap: true,
          physics: const NeverScrollableScrollPhysics(),
          itemCount: widget.playlist.length,
          onReorder: _reorderSounds,
          itemBuilder: (context, index) {
            final sound = widget.playlist[index];
            return _buildSoundTile(sound, index);
          },
        ),
        // ... 
      ],
    );
  }
}
\end{lstlisting}
\end{figure}

\subsection{Menu - inne}
Zakładka „Inne” obsługuje edycję obiektu klasy \texttt{Settings}. Zawiera pole \texttt{TextField} do wprowadzenia opisu treningu, a także licznik czasu przygotowawczego z własnym formatterem cyfr oraz walidacją wartości minimalnej. 

\begin{figure}[H]
\centering
\begin{lstlisting}[caption={Obsługa pola tekstowego z walidacją dla czasu przygotowania}]
TextField(
  key: ValueKey('preparation_${widget.training.hashCode}'),
  controller: preparationController,
  focusNode: preparationFocusNode,
  keyboardType: TextInputType.number,
  textAlign: TextAlign.center,
  inputFormatters: [
    FilteringTextInputFormatter.digitsOnly,
  ],
  onChanged: (value) {
    int newValue = int.tryParse(value) ?? 0;
    setState(() {
      widget.training.settings.preparationDuration = newValue;
    });
  },
  // ...
)
\end{lstlisting}
\end{figure}

Tak jak każdy element graficznego interfejsu użytkownika w aplikacji, edytor jest przetłumaczony z użyciem \texttt{TranslationProvider}, co umożliwia dynamiczne zmiany języka i konsekwentne stosowanie lokalnych etykiet w dialogach, walidatorach czy komunikatach ostrzegawczych. Dzięki temu moduł edytora zachowuje spójność wizualną i logiczną z resztą aplikacji oraz zapewnia użytkownikowi poczucie kontroli nad każdym aspektem treningu bez konieczności przełączania kontekstu.

\section{Importowanie oraz eksportowanie treningów}

Funkcjonalności importu oraz eksportu treningów w aplikacji zostały zaimplementowane w celu zapewnienia użytkownikom łatwej wymiany konfiguracji treningowych pomiędzy różnymi urządzeniami oraz tworzenia kopii zapasowych swoich ustawień pod nieobecność systemu scentralizowanego serwera do persystencji i synchronizacji danych.

Rozwiązanie jest realizowane przez dedykowany serwis {TrainingImportExportService}. Pełni on rolę fasady dla operacji wejści-wyjścia związanych z plikami treningowymi, lecz sam nie zajmuje się logiką biznesową. Logikę biznesową wykonuje na żądanie wcześniej wymienionego serwisu klasa pomocnicza \texttt{TrainingJsonConverter}, która zajmuje się operacjami serializacji obiektów treningów i deseralizacji zrzutów danych. W celu wywołania natywnych, systemowych okien dialogowych wyboru i zapisu plików została wykorzystana biblioteka \texttt{file\_picker} zapewniająca spójne doświadczenie użytkowników niezależnie od platformy (Android/iOS).

Eksport danych możemy wykonać z dwóch miejsc w aplikacji. Pierwszą możliwością jest naciśnięcie przycisku udostępniania dostępnego z poziomu widoku szczegółów treningu (\texttt{TrainingPage}). Można w ten sposób wyeksportować pojedynczy trening okraszony dynamicznie wygenerowaną na podstawie tytułu treningu nazwą pliku. Drugim punktem wywołania eksportu jest zaimplementowany na ekranie głównym (\texttt{HomePage}), przycisk umożliwiający masowy eksport wielu treningów jednocześnie. Użytkownik, korzystając z trybu wyboru, zaznacza poprzez dotknięcie interesujące go pozycje, które następnie są pakowane w zbiorczą strukturę JSON. Proces wybierania został celowo zaprojektowany w sposób zbliżony do zaznaczania plików w managerach plików systemów moiblnych. Plik wynikowy otrzymuje nazwę zawierającą znacznik czasu, co ułatwia katalogowanie kopii zapasowych.

Struktura wyeksportowanego pliku JSON jest kompletnym odzwierciedleniem modelu danych aplikacji. Zawiera ona metadane (tytuł, opis), pełną hierarchię etapów (\texttt{TrainingStage}) wraz z fazami oddechowymi (\texttt{BreathingPhase}) i ich parametrami czasowymi, a także szczegółową konfigurację ustawień (\texttt{Settings}) oraz zależności dźwięków (\texttt{Sounds}). Dzięki temu, wyeksportowany plik jest samowystarczalną jednostką informacji, możliwą do osadzenia w dowolnej innej instancji aplikacji.

Metody parsujące w klasie pomocniczej \texttt{TrainingJsonConverter} zostały zaprojektowane tak, aby rozpoznawać i poprawnie przetwarzać zarówno pojedyncze obiekty treningów, jak i listy obiektów lub struktury opakowane (używane przy eksporcie masowym). Po wczytaniu danych następuje proces walidacji oraz integracji, w ramach którego odtwarzane są powiązania do zasobów dźwiękowych metodą \texttt{updateSounds()}. Poprawnie zweryfikowane treningi są następnie dodawane do lokalnej bazy danych Hive, a interfejs użytkownika jest natychmiastowo odświeżany.

\begin{figure}[H]
\centering
\begin{lstlisting}[caption={Metoda eksportu treningu}, label={code/import_export/export}]
static Future<bool> exportTraining(Training training, {String? fileName}) async {
  try {
    final String defaultFileName = fileName ??
        '${TextUtils.sanitizeFileName(training.title)}_training.json';
    
    final String jsonString = TrainingJsonConverter.toJson(training);
    final Uint8List bytes = Uint8List.fromList(utf8.encode(jsonString));
    
    String? outputPath = await FilePicker.platform.saveFile(
      dialogTitle: TranslationProvider().getTranslation("FilePicker.save_training"),
      fileName: defaultFileName,
      type: FileType.custom,
      allowedExtensions: ['json'],
      bytes: bytes,
    );
    
    return outputPath != null;
  } catch (e) {
    debugPrint('Error during training export: $e');
    return false;
  }
}
\end{lstlisting}
\end{figure}

\begin{figure}[H]
\centering
\begin{lstlisting}[caption={Metoda importu treningów}, label={code/import_export/import}]
static Future<List<Training>?> importTrainings() async {
  try {
    FilePickerResult? result = await FilePicker.platform.pickFiles(
      type: FileType.custom,
      allowedExtensions: ['json'],
      allowMultiple: false,
    );
    
    if (result == null || result.files.isEmpty) {
      return null;
    }
    
    final PlatformFile file = result.files.single;
    String? jsonString;
    
    if (file.bytes != null) {
      jsonString = utf8.decode(file.bytes!);
    } else if (file.path != null) {
      jsonString = await File(file.path!).readAsString();
    }

    if (jsonString == null) return null;

    return TrainingJsonConverter.fromJsonMultiple(jsonString);
  } catch (e) {
    debugPrint('Error during training import: $e');
    return null;
  }
}
\end{lstlisting}
\end{figure}

\section{Przebieg treningu \Hania}
\subsection{Klasa TrainingParser}\label{subsec:TrainingParser}
Klasa \texttt{TrainingParser} powstała w celu przekształcenia hierarchicznych danych treningowych pobranych z lokalnej bazy danych Hive w ciąg występujących po sobie faz, oparty na skonfigurowanym uprzednio przez użytkownika wzorcu oddychania. Jej zadaniem jest zwracanie kolejnych faz do obiektu \texttt{TrainingController}. Dzięki temu logika przełączania faz i powtórzeń jest odseparowana od interfejsu. 

Konstruktor jako parametr przyjmuje obiekt klasy \texttt{Training} i zapisuje do zmiennej \textit{currentTrainingStage} pierwszy etap treningu. Fragment realizujący tą funkcjonalność przedstawiono poniżej w kodzie \ref{code/parser/constructor}.

\begin{figure}[h]
\centering
\begin{lstlisting}[caption={TrainingParser - konstruktor}, label={code/parser/constructor}]
  Training training;
  TrainingStage currentTrainingStage;
  late breathing_phase.BreathingPhase currentBreathingPhase;

  TrainingParser({required this.training})
      : currentTrainingStage = training.trainingStages[0];
\end{lstlisting}
\end{figure}

\newpage
Zadaniem funkcji \texttt{nextInstruction}, przedstawionej na kodzie \ref{code/parser/nextInstruction}, jest zwrócenie danych dotyczącej kolejnej fazy oddechowej w postaci mapy \texttt{Map<String, dynamic>} lub wartości \texttt{null} w~przypadku zakończenia całego treningu. 

W pierwszej kolejności analizowany jest aktualny indeks fazy oddechowej (\textit{breathingPhaseID}). Jeżeli wskazuje on ostatnią fazę w bieżącym etapie, oznacza to zakończenie jednego pełnego cyklu etapu. W takim przypadku indeks fazy jest zerowany (\textit{breathingPhaseID = 0}), a licznik wykonanych powtórzeń w etapie (\textit{doneReps}) zwiększany jest o jeden. Następnie sprawdzana jest liczba wykonanych powtórzeń w odniesieniu do wartości zdefiniowanej w obiekcie etapu (\textit{currentTrainingStage.reps}). W przypadku jej osiągnięcia następuje przejście do kolejnego etapu treningu poprzez inkrementację indeksu \textit{trainingStageID}. Jeżeli po tej operacji indeks ten osiągnie wartość równą liczbie wszystkich etapów w strukturze treningu, funkcja zwraca \textit{null}, sygnalizując zakończenie sesji. W przeciwnym razie wczytywany jest nowy etap (\textit{currentTrainingStage = training.trainingStages[trainingStageID]}), a licznik \textit{doneReps} zostaje zresetowany do zera. Gdy aktualna faza nie była ostatnią w cyklu, indeks \textit{breathingPhaseID} jest jedynie zwiększany o jeden. Po ustaleniu poprawnego indeksu do zmiennej \textit{currentBreathingPhase} przypisywana jest odpowiadająca mu faza oddechowa. Kolejnym etapem jest obliczenie rzeczywistego czasu trwania fazy z uwzględnieniem mechanizmu progresji. 

Na podstawie obliczonego czasu tworzona jest zmienna \textit{progressedBreathingPhase}, w której pole \textit{duration} przyjmuje wartość \textit{durationSeconds}, natomiast pozostałe atrybuty (\textit{breathingPhaseType}, \textit{breathType}, \textit{breathDepth}, \textit{sounds}) są kopiowane z obecnej fazy oddechowej.

Funkcja zwraca mapę zawierającą następujące klucze:
\begin{itemize}
    \item \textit{breathingPhase} - pełny obiekt fazy,
    \item \textit{remainingTime} - czas trwania fazy wyrażony w milisekundach,
    \item \textit{trainingStageName} - nazwę aktualnego etapu treningu.
\end{itemize}

W ten sposób \textit{nextInstruction()} pełni rolę centralnego mechanizmu sterującego przebiegiem treningu oddechowego, zapewniając poprawne przechodzenie pomiędzy fazami i etapami oraz automatyczne zwiększanie trudności zgodnie z zaimplementowanym modelem progresji liniowej.

\newpage
\begin{figure}[h]
\centering
\begin{lstlisting}[caption={TrainingParser - pobranie instrukcji}, label={code/parser/nextInstruction}]
  Map<String, dynamic>? nextInstruction() {
    if (breathingPhaseID == currentTrainingStage.breathingPhases.length - 1) {
      breathingPhaseID = 0;
      doneReps++;

      if (doneReps == currentTrainingStage.reps) {
        trainingStageID++;
        if (trainingStageID == training.trainingStages.length) {
          return null;
        } else {
          currentTrainingStage = training.trainingStages[trainingStageID];
          doneReps = 0;
        }
      }
    } else {
      breathingPhaseID++;
    }

    currentBreathingPhase = currentTrainingStage.breathingPhases[breathingPhaseID];

    double durationSeconds = currentBreathingPhase.duration + (currentTrainingStage.increment * doneReps);

    final progressedBreathingPhase = breathing_phase.BreathingPhase(
      duration: durationSeconds,
      breathingPhaseType: currentBreathingPhase.breathingPhaseType,
      breathType: currentBreathingPhase.breathType,
      breathDepth: currentBreathingPhase.breathDepth,
      sounds: currentBreathingPhase.sounds,
    );

    return {
      "breathingPhase": progressedBreathingPhase,
      "remainingTime": (durationSeconds * 1000).truncate(),
      "trainingStageName": currentTrainingStage.name,
    };
  }
\end{lstlisting}
\end{figure}

\subsection{TrainingController} \label{subsec:TrainingController}
\subsection{AnimatedCircle} \label{subsec:AnimatedCircle}
Komponent \texttt{AnimatedCircle} odpowiada za wizualizację przebiegu fazy oddechowej w postaci animowanego koła. Wdech powoduje zwiększanie jego promienia, wydech - zmniejszanie, natomiast fazy retencji i regeneracji utrzymują stały rozmiar.

Obiekt przyjmuje dwa parametry - obiekt typu \texttt{BreathingPhase?}, który reprezentuje aktualną fazę oddechową bądź wartość null w przypadku zakończenia treningu oraz obiekt typu \texttt{bool} \textit{isPaused}, reprezentujący stan wstrzymania treningu. Framgment tworzenia klasy przedstawiony został w kodzie \ref{code/circle/constructor}.

\begin{figure}[h]
\centering
\begin{lstlisting}[caption={AnimatedCircle - konstruktor}, label={code/circle/constructor}]
  final breathing_phase.BreathingPhase? breathingPhase;
  final bool isPaused;

  const AnimatedCircle({super.key, required this.breathingPhase, required this.isPaused});
\end{lstlisting}
\end{figure}

Po utworzeniu obiektu na początku zostaje obliczona wartość początkowa czasu trwania animacji, która zapisywana jest w zmiennej \textit{duration} na podstawie czasu trwania danej fazy. Następnie inicjowany jest kontroler animacji \textit{controller} oraz animacja zmiany promienia koła \textit{circleAnimation}. Kontrolwe ustawiany jest na stan początkowy (\textit{controller.value=0.0}). Jeżeli trening nie jest wstrzymany i dostępna jest faza oddechowa, uruchamiana jest odpowiednia animacja - rosnąca dla wdechu lub malejąca dla wydechu. Fragment funkcji init przedstawiono w kodzie \ref{code/circle/init}.

\begin{figure}[h]
\centering
\begin{lstlisting}[caption={AnimatedCircle - fragment funkcji init}, label={code/circle/init}]
  @override
  void initState() {
    super.initState();

    duration = widget.breathingPhase == null ? 0 : (widget.breathingPhase!.duration * 1000).toInt();

    _controller = AnimationController(
      duration: Duration(milliseconds: duration),
      vsync: this,
    );

    _circleAnimation = Tween<double>(begin: 125.0, end: 300.0).animate(
      CurvedAnimation(parent: _controller, curve: Curves.easeInOut),
    );
    _controller.duration = Duration(milliseconds: duration);

    _controller.value = 0.0;

    if (!widget.isPaused && widget.breathingPhase != null) {
      if (widget.breathingPhase!.breathingPhaseType == breathing_phase.BreathingPhaseType.inhale) {
        _controller.forward(from: 0.0);
      } else if (widget.breathingPhase!.breathingPhaseType == breathing_phase.BreathingPhaseType.exhale) {
        _controller.reverse(from: 1.0);
      }
    }
  }
\end{lstlisting}
\end{figure}

Zachowanie koła zależne jest od jego poprzedniego stanu i zmiany parametrów wejściowych. Jeżeli faza oddechowa sie zmieniła względem ostatniego stanu następuje reakcja zmiany animacji. Ponownie obliczany jest czas trwania animacji (zmienna \textit{duration}) i ustawiane zostaje poprawne zachowanie - wzrost promienia dla wdechu i jego zmniejszenie dla wydechu. Aktualizację stanu przedstawiono w kodzie \ref{code/circle/update}

\begin{figure}[h]
\centering
\begin{lstlisting}[caption={AnimatedCircle - aktualizacja stanu}, label={code/circle/update}]
    if (widget.breathingPhase != oldWidget.breathingPhase && widget.breathingPhase != null) {
      log("${widget.breathingPhase?.breathingPhaseType.name}");

      duration = widget.breathingPhase == null ? 0 : (widget.breathingPhase!.duration * 1000).toInt();
      _controller.duration = Duration(milliseconds: duration);

      if (!widget.isPaused && widget.breathingPhase != null) {
        if (widget.breathingPhase!.breathingPhaseType == breathing_phase.BreathingPhaseType.inhale) {
          _controller.forward(from: 0.0);
        } else if (widget.breathingPhase!.breathingPhaseType == breathing_phase.BreathingPhaseType.exhale) {
          _controller.reverse(from: 1.0);
        }
      } else {
        _controller.stop();
      }
    }
\end{lstlisting}
\end{figure}

Koło \texttt{AnimatedCircle} reaguje również na zmianę parametru \textit{isPaused}, która określa czy trening został zatrzymany. Jeśli tak, animacja zostaje wstrzymana, w innym wypadku jeśli trening był zatrzymany i zostawł wznowiony, animacja zostaje kontynuowana w kierunku wynikającym z~bieżącej fazy. Fragment ten przedstawiono w kodzie \ref{code/circle/pause}.

\begin{figure}[h]
\centering
\begin{lstlisting}[caption={AnimatedCircle - reakcja na pauzę i wznowienie}, label={code/circle/pause}]
if (widget.isPaused && !oldWidget.isPaused) {
      _controller.stop();
    } else if (!widget.isPaused && oldWidget.isPaused) {
      if (widget.breathingPhase != null) {
        if (widget.breathingPhase!.breathingPhaseType == breathing_phase.BreathingPhaseType.inhale) {
          _controller.forward();
        } else if (widget.breathingPhase!.breathingPhaseType == breathing_phase.BreathingPhaseType.exhale) {
          _controller.reverse();
        }
      }
    }
\end{lstlisting}
\end{figure}

Dodatkowo utworzony został obiekt łączący \texttt{AnimatedCircle} oraz dwa koła statyczne typu \texttt{Container} wizualizujące maksymalną i minimalną wartość promienia koła \texttt{AnimatedCircle}, by umożliwić użytkownikowi lepiej ocenić przebieg wdechu i wydechu.

\subsection{InstructionSlider}
\texttt{InstructionSlider} to animowany komponent odpowiadający za wizualizację ciągu instrukcji dla użytkownika w postaci trzech kafelków reprezentujących:
\begin{itemize}
    \item poprzednią fazę oddechową,
    \item obecną fazę oddechową,
    \item nadchodzącą fazę oddechową.
\end{itemize}
Gdy zachodzi zmiana fazy, cała lista przesuwa się w lewo, a na końcu pojawia się nowy kafelek z~kolejną instrukcją.

Obiekt przyjmuje trzy parametry - \textit{preparationTime}, czyli czas trwania fazy przygotowania przed treningiem, kolejkę \textit{breathingPhasesQueue}, która jest ciągiem faz oddechowych w treningu oraz \textit{change}, która jest wyznacznikiem zmiany obecnie trwającej fazy. Framgment tworzenia klasy przedstawiony został w kodzie \ref{code/slider/constructor}.

\begin{figure}[h] 
\centering
\begin{lstlisting}[caption={InstructionSlider - konstruktor}, label={code/slider/constructor}]
double preparationTime;
Queue<breathing_phase.BreathingPhase?> breathingPhasesQueue = Queue<breathing_phase.BreathingPhase?>();
int change; 

InstructionSlider({super.key,required this.preparationTime,  required this.breathingPhasesQueue, required this.change});
\end{lstlisting}
\end{figure}

Podczas inicjalizacji obiekt tworzy pierwszy kafelek reprezentujący fazę przygotowania. Następnie dodawane są dwie fazy z kolejki faz treningu w tej samej formie . Każdy kafelek przechowuje tekst w postaci instrukcji dla użytkowanika i informację, na której pozycji ma się znaleźć. Fragment dodawania obiektów przedstawiono w kodzie \ref{code/slider/init}.

\begin{figure}[h] 
\centering
\begin{lstlisting}[caption={InstructionSlider - fragment funkcji init}, label={code/slider/init}]
_blocks.add(
  InstructionBlock(
    text: translationProvider.getTranslation(
      "BreathingPage.InstructionSlider.get_ready_block_text") + "\n${widget.preparationTime} s", 
    position: 0.0)
);

addNewBreathingPhase(widget.breathingPhasesQueue.elementAt(1));
addNewBreathingPhase(widget.breathingPhasesQueue.elementAt(2));
\end{lstlisting}
\end{figure}

\texttt{InstructionSlider} korzysta z \texttt{AnimationController}, który przy każdej zmianie fazy przesuwa wszystkie kafelki o jedną pozycję w lewo. Każdy kafelek ma przypisaną pozycję z przedziału od -2 do 2. Wartości skrajne (-2 oraz 2) znajdują się poza obszarem widocznym dla użytkownika i służą jedynie do zapewnienia płynniejszej i estetczniejszej animacji. W centrum ekranu znajdują się trzy środkowe pozycje: -1 odpowiada fazie poprzedniej, 0 fazie bieżącej, a 1 fazie nadchodzącej. Gdy kafelek na pozycji -2 ma zostać przesunięty zostaje usunięty w celach optymalizacyjnych. Dokładną implementację animacji przedstawiono w~kodzie \ref{code/slider/animation}. Dodatkowo wykorzystywany jest mechanizm skalowania środkowego kafelka w celu wizualnego podkreślenia aktualnie trwającej fazy.

\begin{figure}[h] 
\centering
\begin{lstlisting}[caption={InstructionSlider - animacja}, label={code/slider/animation}]
_controller = AnimationController(vsync: this, duration: duration);
_animation = Tween<double>(begin: 0.0, end: -1.0).animate(
  CurvedAnimation(parent: _controller, curve: Curves.easeInOut),
);
_controller.addStatusListener((status) {
  if (status == AnimationStatus.completed) {
    _controller.reset();
    setState(() {
      final removed = _blocks.where((b) => b.position <= -2).toList();
      _blocks.removeWhere((b) => b.position <= -2);
      _blocks.forEach((b) => b.position += _animation.value); 
      _blocks.forEach((b) => b.position -= 1);
    });
  }
});
\end{lstlisting}
\end{figure}

Każda zmiana fazy powoduje uruchomienie aktualizacji stanu komponentu. Mechanizm opiera się na porównaniu poprzedniej i bieżącej wartości parametru \textit{change}. Dzięki temu komponent wykrywa moment rozpoczęcia nowej fazy. Obliczany jest wtedy czas trwania nowej fazy, wywoływana jest animacja oraz dodawany nowy kafelek z instrukcją. Funkcję aktualizacji przedstawiono w kodzie \ref{code/slider/update}.

\begin{figure}[h] 
\centering
\begin{lstlisting}[caption={InstructionSlider - aktualizacja stanu}, label={code/slider/update}]
if(oldWidget.change != widget.change) {
  final int phaseDuration = (widget.breathingPhasesQueue.elementAt(0)?.duration.toInt() != null)
    ? (widget.breathingPhasesQueue.elementAt(0)!.duration * 1000).toInt() - 50 
    : 400;
  _controller.duration = Duration(milliseconds: min(phaseDuration,400));
  _controller.forward();
  if(_blocks.last.text!=translationProvider.getTranslation(
    "BreathingPage.InstructionSlider.ending_tile_text")) {
    addNewBreathingPhase(widget.breathingPhasesQueue.elementAt(2));
  }
}
\end{lstlisting}
\end{figure}

\section{Dźwięki \Jakub} \label{sec:Sounds}

Wysoki stopień konfigurowalności warstwy audio przez użytkownika wymusił dekompozycję logiki biznesowej aplikacji na klasy odpowiedzialne za poszczególne przypadki użycia, postępując zgodnie z zasadą pojedynczej odpowiedzialności. 


\begin{figure}[H]
  \centering
  \includegraphics[width=\textwidth]{diagramy/soundManagers_diagram.png}
  \caption{Diagram klas dźwiękowych}
\end{figure}

\subsection{SoundManager} \label{subsec:SoundManager}
Centralnym elementem modułu audio jest klasa \texttt{SoundManager}. Ze względu na konieczność zapewnienia globalnego punktu dostępu do listy załadowanych plików oraz potrzebę współdzielenia jednej instancji obiektu przez wiele komponentów aplikacji, zastosowano w niej wzorzec projektowy Singleton. W polach wspomnianej instancji przechowywane są listy dźwięków krótkich, jak i długich. Rozróżnienie na typy dźwięków było niezbędne ze względu na odmienne ładowanie i odtwarzanie plików. Krótkie efekty dźwiękowe wymagają minimalnej latencji (czasu reakcji) przy odtwarzaniu, natomiast długie ścieżki dźwiękowe, charakteryzujące się rzadszą rotacją i mniejszą dynamiką zmian, nie podlegają tak rygorystycznym wymogom czasowym przy inicjalizacji. Do osiągnięcia tego wymagania należało zastosować oddzielne konfiguracje odtwarzaczy dźwięków (komponentów klasy \texttt{AudioPlayer}) — low latency mode (tryb minimalnego czasu reakcji) lub media player mode (tryb odtwarzania media). Mimo zastosowania tych trybów wysoki narzut obliczeniowy związany z  częstą inicjalizacją odtwarzaczy uniemożliwił odtwarzanie sygnałów dźwiękowych z wymaganą precyzją. Ograniczenie to zmusiło nas do wprowadzenia mechanizmu puli obiektów — \texttt{AudioPlayerPool} \ref{subsec:AudioPlayerPool}.

 \todoJakub{Rozpisz poniższe}
- fade in/out
- ustawienie by dźwięki mogły lecieć razem

\subsection{SingleSoundManager} \label{subsec:SingleSoundManager}
\subsection{PlaylistManager} \label{subsec:PlaylistManager}
\subsection{AudioPlayerPool} \label{subsec:AudioPlayerPool}
\subsection{textToSpeech} \label{subsec:TTS}

\section{Języki} \label{sec:Languages}

\section{Baza danych \Karol}
W celu zapewnienia persystencji danych użytkownika w naszej aplikacji mobilnej zaimplementowaliśmy lokalną bazę danych. Zważywszy na wybrane wcześniej środowisko Flutter, wymaganie wspierania wielu platform oraz wysoką gęstość zapisu danych, wybraliśmy napisaną w języku Dart nierelacyjną bazę Hive w wersji 2.2.3. Jest to lekka baza typu \textit{key-value store}, która przechowuje dane w formacie binarnym bezpośrednio na urządzeniu użytkownika, bez konieczności łączenia się z serwerem zewnętrznym.

\subsection{Architektura i generowanie kodu}
Hive wykorzystuje mechanizm generowania kodu adapterów zajmujących się niskopoziomowymi operacjami na plikach bazy danych. Dzięki bardzo dobrej integracji z językiem Dart umożliwia łatwe odwzorowanie klas w bazie danych poprzez adnotacje \texttt{@HiveType()} oraz \texttt{@HiveField()}. Adnotacja \texttt{@HiveType(typeId: n)} oznacza klasę jako typ możliwy do serializacji, gdzie \texttt{typeId} musi być unikalnym identyfikatorem liczbowym. Adnotacja \texttt{@HiveField(n)} oznacza poszczególne pola klasy, które mają być zapisywane do bazy.

Listing \ref{code/hive/settings} przedstawia przykładową klasę \texttt{Settings} z adnotacjami Hive:

\begin{lstlisting}[caption={Klasa Settings z adnotacjami Hive}, label={code/hive/settings}]
import 'package:hive_flutter/hive_flutter.dart';

part 'Settings.g.dart';

@HiveType(typeId: 8)
class Settings {
  
  @HiveField(0)
  int preparationDuration = 3;
  
  @HiveField(1)
  int endingDuration = 5;

  @HiveField(2)
  bool binauralBeatsEnabled = false;
  
  @HiveField(3)
  double binauralLeftFrequency = 200.0;
  
  @HiveField(4)
  double binauralRightFrequency = 210.0;
}
\end{lstlisting}

Po oznaczeniu klasy danych adnotacjami i uruchomieniu polecenia \texttt{flutter pub run build\_runner build}, generator automatycznie tworzy pliki z rozszerzeniem \texttt{.g.dart} zawierające klasy \texttt{TypeAdapter}. Adaptery te implementują logikę serializacji i deserializacji obiektów do formatu binarnego:
\begin{itemize}
    \item metoda \texttt{read()} -- odczytuje dane binarne z bazy i odtwarza obiekt Dart,
    \item metoda \texttt{write()} -- przetwarza obiekt Dart na ciąg bajtów do zapisu.
\end{itemize}

Listing \ref{code/hive/adapter} przedstawia fragment automatycznie wygenerowanego adaptera dla klasy \texttt{Settings}:

\begin{lstlisting}[caption={Wygenerowany SettingsAdapter}, label={code/hive/adapter}]
part of 'Settings.dart';

class SettingsAdapter extends TypeAdapter<Settings> {
  @override
  final int typeId = 8;

  @override
  Settings read(BinaryReader reader) {
    final numOfFields = reader.readByte();
    final fields = <int, dynamic>{
      for (int i = 0; i < numOfFields; i++) 
        reader.readByte(): reader.read(),
    };
    return Settings()
      ..preparationDuration = fields[0] as int
      ..endingDuration = fields[1] as int
      ..binauralBeatsEnabled = fields[2] as bool
      ..binauralLeftFrequency = fields[3] as double
      ..binauralRightFrequency = fields[4] as double;
  }

  @override
  void write(BinaryWriter writer, Settings obj) {
    writer
      ..writeByte(5)
      ..writeByte(0)
      ..write(obj.preparationDuration)
      ..writeByte(1)
      ..write(obj.endingDuration)
      ..writeByte(2)
      ..write(obj.binauralBeatsEnabled)
      ..writeByte(3)
      ..write(obj.binauralLeftFrequency)
      ..writeByte(4)
      ..write(obj.binauralRightFrequency);
  }


  @override
  int get hashCode => typeId.hashCode;

  @override
  bool operator ==(Object other) =>
      identical(this, other) ||
      other is SettingsAdapter &&
          runtimeType == other.runtimeType &&
          typeId == other.typeId;
}
\end{lstlisting}

\subsection{Model danych}
W aplikacji zdefiniowaliśmy następujące typy danych przechowywane w bazie Hive:
\begin{itemize}
    \item \texttt{Training} (typeId: 1) -- kompletny trening oddechowy zawierający tytuł, opis, listę etapów, ustawienia dźwięków oraz konfigurację,
    \item \texttt{TrainingStage} (typeId: 2) -- pojedynczy etap treningu z liczbą powtórzeń, przyrostem czasowym oraz listą faz oddechowych,
    \item \texttt{BreathingPhaseType} (typeId: 3) -- typ wyliczeniowy określający rodzaj fazy oddechowej (wdech, wydech, zatrzymanie, regeneracja),
    \item \texttt{BreathingPhase} (typeId: 4) -- faza oddechowa z określonym czasem trwania, typem i przypisanymi dźwiękami,
    \item \texttt{BreathingPhaseIncrementType} (typeId: 5) -- typ wyliczeniowy określający sposób inkrementacji czasu (procentowy lub stała wartość),
    \item \texttt{BreathingPhaseIncrement} (typeId: 6) -- konfiguracja przyrostu czasowego dla poszczególnych faz,
    \item \texttt{Sounds} (typeId: 7) -- kontener przechowujący kompletne ustawienia dźwiękowe dla treningu,
    \item \texttt{Settings} (typeId: 8) -- globalne ustawienia treningu, w tym konfiguracja dudnień binauralnych,
    \item \texttt{BreathingPhaseSounds} (typeId: 9) -- zestaw dźwięków przypisanych do konkretnej fazy oddechowej (dźwięk tła i wskazówka rozpoczęcia),
    \item \texttt{SoundAsset} (typeId: 10) -- reprezentacja zasobu dźwiękowego z informacją o ścieżce pliku i jego typie,
    \item \texttt{SoundType} (typeId: 11) -- typ wyliczeniowy kategoryzujący dźwięki (np. głos, melodia, sygnał),
    \item \texttt{SoundScope} (typeId: 12) -- typ wyliczeniowy definiujący zakres obowiązywania tła muzycznego (np. globalnie, per etap).
\end{itemize}

\subsection{Organizacja danych w Box'ach}
Hive organizuje dane w strukturach zwanych \textit{Box}'ami, które odpowiadają tabelom w bazach relacyjnych lub kolekcjom w bazach dokumentowych. Każdy Box przechowuje pary klucz-wartość i musi zostać otwarty przed użyciem. W naszej aplikacji zdefiniowaliśmy cztery Box'y:
\begin{itemize}
    \item \texttt{respire} -- główny Box przechowujący presety treningowe oraz ich wersję migracyjną,
    \item \texttt{userShortSounds} -- Box na krótkie dźwięki dodane przez użytkownika,
    \item \texttt{userLongSounds} -- Box na długie dźwięki tła dodane przez użytkownika,
    \item \texttt{userCountingSounds} -- Box na dźwięki odliczania dodane przez użytkownika.
\end{itemize}

\subsection{Inicjalizacja bazy danych}
Inicjalizacja bazy danych odbywa się w funkcji \texttt{initialize()} wywoływanej przy starcie aplikacji. Proces ten obejmuje:
\begin{enumerate}
    \item Inicjalizację silnika bazy danych -- \texttt{Hive.initFlutter()}
    \item Rejestrację wszystkich adapterów typów -- \texttt{Hive.registerAdapter()}
    \item Otwarcie wszystkich Box'ów -- \texttt{Hive.openBox()}.
\end{enumerate}

\begin{lstlisting}[caption={Inicjalizacja bazy danych Hive}, label={code/hive/init}]
Future<void> initialize() async {
  await Hive.initFlutter();
  
  Hive.registerAdapter(BreathingPhaseAdapter());
  Hive.registerAdapter(BreathingPhaseTypeAdapter());
  Hive.registerAdapter(TrainingStageAdapter());
  Hive.registerAdapter(TrainingAdapter());
  Hive.registerAdapter(SoundAssetAdapter());
  Hive.registerAdapter(SoundsAdapter());
  Hive.registerAdapter(SettingsAdapter());
  Hive.registerAdapter(BreathingPhaseSoundsAdapter());
  
  await Hive.openBox('respire');
  await Hive.openBox('userShortSounds');
  await Hive.openBox('userLongSounds');
  await Hive.openBox('userCountingSounds');
}
\end{lstlisting}

Operacje na danych wykonywane są poprzez metody Box'a: \texttt{put(key, value)} do zapisu, \texttt{get(key)} do odczytu oraz \texttt{delete(key)} do usuwania. Przykładowe operacje na danych - w tym przypadku ładowania treningów i dodawanie ich do listy presetów - pokazano w kodzie \ref{code/hive/operations}.

\begin{lstlisting}[caption={Operacje na danych w Hive}, label={code/hive/operations}]
final _box = Hive.box('respire');

void loadData() {
  final stored = _box.get('presets');
  if (stored != null) {
    presetList = List<Training>.from(stored);
  }
}

void updateDataBase() {
  _box.put('presets', presetList);
}
\end{lstlisting}

Dzięki generowanym adapterom Hive automatycznie serializuje złożone obiekty wraz z ich zagnieżdżonymi strukturami, co umożliwia zapisanie całego drzewa obiektu \texttt{Training} wraz ze wszystkimi etapami, fazami i dźwiękami w jednej operacji.