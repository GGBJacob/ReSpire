\chapter{Instrukcja użytkowania}
Celem tego rozdziału jest przedstawienie zbioru funkcji aplikacji ReSpire. W kolejnych podrozdziałach opisano sposób instalacji aplikacji oraz wszystkie ekrany aplikacji zaprezentowane poprzez zrzuty ekranu.

\section{Instalacja aplikacji \Ola}
Na urządzenie mobilne z systemem Android należy pobrać plik instalacyjny aplikacji w formacie \textit{.apk}. Następnie należy uruchomić pobrany plik poprzez ppojedyncze kliknięcie i postępować zgodnie z instrukcjami na ekranie. Na ekranie telefonu może się wtedy pojawić okno z zapytaniem o zezwolenie na instalację aplikacji spoza Sklepu Google Play, jeśli to ustawienie jest wyłączone. Należy wtedy kliknąć przycisk \textit{Ustawienia}, a następnie zezwolić na instalację przesuwając przycisk w prawo. Można tego dokonać również poprzez wejście w odpowiednią zakładkę bezpośrednio z poziomu aplikacji ustawień (Aplikacje -> Specjalny dostęp do aplikacji -> Instalowanie nieznanych aplikacji -> Files by Google lub inny eksplorator plików zainstalowany na urządzeniu). Po pojawieniu się okna z zapytaniem o instalację należy kliknąć przycisk \textit{Zainstaluj} i poczekać na zakończenie instalacji. Aplikacja powinna następnie pojawić się na liście aplikacji na urządzeniu i być gotowa do uruchomienia.

\section{Strona główna}
Ekran główny aplikacji składa się z dwóch sekcji: u góry znajduje się pasek z opcjami, a~resztę ekranu zajmują kafelki z~dostępnymi treningami, zgodnie z~RYSUNEK. Na dole ekranu widoczna jest animacja fal.

Na środku paska znajduje się logo aplikacji. Po lewej stronie umieszczona jest ikona symbolizująca tryb zaznaczania, który jest aktywowany po kliknięciu. Tryb zmienia wygląd paska - po lewej stronie znajduje się symbol krzyżyka pozwalający opuścić tryb. Obok znajduje się informacja, że tryb zaznaczania jest aktywowany. Użytkownik może wybrać dowolne treningi z listy poprzez kliknięcie kafelków z~nazwami treningów - zaznaczone elementy oznaczone są ciemną, pogrubioną obwódką lub kilknąć przycisk \textit{Zaznacz wszystkie}, który zaznaczy automatycznie wszystkie dostępne treningi. Po wybraniu co najmniej jednego treningu pojawia się informacja o~ilości wybranych treningów. Treningi można także odznaczyć, jeśli użytkownik zmieni zdanie. Jeśli wybór treningów do eksportu jest satysfakcjonujący należy nacisnąć ikonę z~prawej strony paska. Po wybraniu miejsca eksportu w~eksploratorze plików, który się otworzy, opcjonalnie zmianie nazwy pliku i~kliknięciu przycisku \textit{Zapisz} plik w~formacie JSON zostanie zapisany na urządzeniu. W aplikacji wyświetli się także odpowiedni komunikat mówiący o~tym, czy zapis się powiódł. (RYSUNEK). W przypadku eksportu kilku treningów, zostaną one zapisane w jednym pliku. Opcja eksportu treningu dostępna jest także z poziomu strony treningu, jednak w przeciwności do strony głównej, nie pozwala na grupowy eksport, a jedynie indywidualnego treningu. 

W prawej części paska znajdują się dwie ikony: koło zębate przenoszące użytkownika na stronę ustawień oraz symbol ze~strzałką w~dół umożliwiający import zapisanych treningów. Po klikniknięciu w~ikonę importu następuje przeniesienie do eksploratora plików na urządzeniu z~domyślnie otworzonym folderem \textit{Pobrane} na urządzeniu, skąd można wybrać plik z~zapisanym wcześniej treningiem/treningami.

Na stronie widoczna jest lista treningów użytkownika w postaci klikalnych kafelków z~nazwą treningu i~ozdobnym detalem symbolizującym podmuch powietrza. Naciśnięcie kafelka przenosi na stronę treningu, która zawiera szczegóły, możliwość edycji, usunięcia, eskportu czy rozpoczęcia treningu. Domyślnie, po zainstalowaniu aplikacji, dostępne są 3 predefiniowane treningi, widoczne na RYSUNEK. W przypadku, gdy użytkownik chce dodać swój trening, może to zrobić klikając ikonę plusa w~białym okręgu z~niebieskim cieniem. Zostanie on~wtedy przeniesiony na~stronę konfiguracji treningu, która opisana jest w późniejszym podrozdziale LINK. Po dodaniu indywidualnie skonfigurowanego treningu wyświetli się on na liście.

\section{Strona szczegółów treningu}
Na stronę można przejść, klikając w kafelek z wybranym treningiem na stronie głównej. Na pasku u góry znajduje się tytuł treningu oraz strzałka umożliwiająca powrót na stronę główną. Niżej znajdują się klikalne ikony pozwalające na akcje związane z treningiem. Z lewej strony umieszczona została opcja eksportu, która działa analogicznie do strony głównej, z tą różnicą, że eksportujemy tylko jeden trening. Po prawej natomiast znajduje się opcja usunięcia treningu (oznaczona symbolem kosza na śmieci), a także możliość edycji treningu (symbol ołówka). Przy akcji usuwania wyświetla się okno z prośbą o jej potwierdzenie, co zapobiega przypadkowemu usunięciu treningu. Natomiast po wybraniu opcji edycji aplikacja przenosi użytkownika do opisywanej w kolejnym podrozdziale strony. LINK

Niżej znajduje się sekcja opisu treningu. Można go edytować lub dodać na stronie konfiguracji treningu. Predefiniowane treningi zawierają opisy, jednak nie jest to element wymagany.

Następnym elementem jest \textit{Przegląd treningu}, który po naciśnięciu strzałki po prawej stronie rozwija się, pokazując wszystkie etapy treningu wraz z ich nazwami, fazami oddechowymi i długościami ich trwania, a także liczbę powtórzeń oraz przyrost. Dzięki takiemu podsumowaniu użytkownik może łatwo przejrzeć strukturę treningu.

Na stronie znajduje się również przycisk \textit{Rozpocznij trening}, który przenosi użytkownika na stronę treningu oddechowego, rozpoczynając ćwiczenie.

Dodatkowym elementem strony jest animacja delikatnie falującej łódki.

\section{Strona konfiguracji treningu}
U góry strony konfiguracji treningu, na pasku, umieszczona jest nazwa treningu (przy tworzeniu nowego treningu jest ona nadawana automatycznie), strzałka umożliwiająca powrót do strony ze szczegółami treningu oraz ikona ołówka, po kliknięciu której wyświetla się okno edycji nazwy treningu. Po wprowadzeniu wybranej nazwy należy ją zatwierdzić przyciskiem “Zapisz”, można także wyjść bez zapisu po kliknięciu “Anuluj”. Długość nazwy treningu ograniczona jest do 15 znaków, po osiągnięciu tego limitu dalsze znaki nie będą wpisywane w pole edycji. Na dole, po jego prawej stronie, znajduje się licznik znaków, tak aby użytkownik wiedział, ile znaków pozostało jeszcze do wykorzystania. 

Konfigurator składa się z trzech zakładek: \textit{Trening}, \textit{Dźwięki} oraz \textit{Inne}, tak jak na widocznym RYSUNEK.

Pierwsza z zakładek - \textit{Trening} - służy do tworzenia struktury nowego treningu lub edycji istniejącego. Przyciśnięcie przycisku \textit{Dodaj etap treningu} umożliwia użytkownikowi utworzenie i dołączenie nowego etapu do treningu, co skutkuje pojawieniem się na ekranie kafelka reprezentującego uwtorzony etap pod istniejącymi etapami. Każdy kafelek zawiera nazwę etapu, która jest dodawana domyślnie przy jego tworzeniu. Może ona zostać edytowana poprzez naciśnięcie na pole z nazwą znajdującą się w zaokrąglonej ramce, a następnie wprowadzenie zmian przy pomocy klawiatury, która pokaże się na ekranie. Nazwa etapu jest ograniczona do 25 znaków. Po osiągnięciu limitu, użytkownik nie będzie miał możliwości wpisania więcej znaków. Licznik znaków wyświetlany jest pod polem z nazwą. Usunięcie etapu jest możliwe poprzez kliknięcie ikony kosza na śmieci znajdującej się po prawej stronie pola z nazwą treningu, a następnie potwierdzenie wykonania akcji klikając przycisk \textit{Usuń} w okienku, które się pokaże. Usuwanie można także anulować klikając przycisk \textit{Anuluj}. Poniżej znajdują się pola dedykowane liczbie powtórzeń danego etapu oraz przyrostowi wyrażonemu w sekundach. Liczba powtórzeń definiuje, ile razy po sobie będzie odtwarzany dany etap, natomiast przyrost - o ile sekund będzie dłuższa każda faza oddechowa w kolejnych powtórzeniach. Wartości można edytować poprzez naciśnięcie ikon plusa / minusa (odpowiednio zwiększenie / zmniejszenie wartości) lub klikając w pole z aktualną liczbą, co spowoduje pokazanie klawiatury na ekranie i umożliwi użytkownikowi wpisanie wybranej wartości. W przypadku używania przycisków plusa lub minusa dla powtórzeń wartość będzie się zmieniać o 1, natomiast dla przyrostu - o 0.1 sekundy w przedziale od 0 do 1 sekundy lub o 1 sekundę dla wartości większych niż 1 sekunda. Za pomocą przycisku \textit{Dodaj fazę oddechu} użytkownik może dodać fazy oddechu do etapu, które wyświetlane są w postaci listy kafelków. Można usunąć fazę poprzez kliknięcie ikony kosza na śmieci umiejszonego po prawej stronie kafelka, a następnie potwierdzenia akcji za pomocą przycisku \textit{Usuń}. Można także zmienić kolejność faz poprzez naciśnięcie i przytrzymanie symbolu dwóch równoległych linii umieszczonego po lewej stronie 

W zakładce \textit{Inne} znajdują się pozostałe ustawienia - możliwe jest dodanie nowego lub modyfikacja istniejącego opisu treningu oraz ustawianie długości trwania przygotowania. Należy to zrobić używając przycisków plusa / minusa (odpowiednio zwiększenie / zmniejszenie wartości) lub klikając w okno z aktualnie ustawioną liczbą, a następnie wpisując wybraną liczbę z klawiatury, która się pokaże.

Wszystkie zmiany wprowadzone do konfiguratora zostaną zapisane automatycznie po jego opuszczeniu (czyli kliknięciu strzałki w lewym górnym rogu paska strony). W przypadku próby zapisania treningu z pustymi etapami użytkownik otrzyma komunikat widoczny na RYSUNEK.

\section{Strona treningu oddechowego}
Podczas wczytywania strony treningu oddechowego może się pojawić komunikat informujący o ładowaniu dźwięków. U góry strony widoczny jest tytuł odtworzonego treningu, a także strzałka umożliwiająca powrót do strony szczegółów treningu oraz po prawej stronie przycisk pauzy, jeśli użytkownik chce wstrzymać trening. W celu wznowienia treningu należy nacisnąć ikonę odtwarzania, która pojawi się w miejsce ikony pauzy lub napis \textit{Wznów} na środku opisanej poniżej animacji.

Podczas treningu wyświetlana jest nazwa aktualnego etapu treningu, a poniżej niej - informacja na którym etapie z ilu łącznie jest użytkownik. Na ekranie znajduje się także karuzela z kafelkami, które przesuwają się wraz z przbiegiem treningu. Kafelki zawierają nazwę kroku (fazy oddechowej, rozpoczęcia lub zakończenia treningu) oraz czas jego trwania. Centralny, największy kafelek symbolizuje obecnie trwający krok, kafelek na lewo od niego - poprzedni krok, a kafelek na prawo od centralnego - kolejny krok. Dzięki temu użytkownik może śledzić przebieg treningu oraz przygotować się na nadchodzący krok. Poniżej karuzeli znajduje się licznik wszystkich kroków, co pozwala mniej więcej zorientować się, jak daleko jest od początku treningu.

Głównym elementem strony jest animacja koła, które zmienia się zgodnie z przebiegiem treningu. Podczas wdechu koło powiększa się, a podczas wydechu - pomniejsza. W czasie trwania fazy regneracji lub wstrzymania koło jest natomiast statyczne. Obrazuje to użytkownikowi, jaką fazę oddechową powinien obecnie wykonywać. Na środku animacji znajduje się także licznik czasu przeznaczonego na dany krok.

Jeśli użytkownik kliknie strzałkę powrotu do strony szczegółów treningu, zostanie zapytany o powterdzenie swojej akcji, w celu zapobiegnięcia przypadkowemu opszczeniu treningu.

\section{Strona ustawień}
Strona ustawień zawiera dwie sekcje: wybór języka aplikacji oraz notatkę o ReSpire informującą, jaki jest cel aplikacji. W celu zmiany języka aplikacji należy nacisnąć strzałkę obok informacji o obecnie ustawionym języku, a następnie dokonać wyboru poprzez kliknięcie jednej z dwóch opcji - języka angielskiego lub języka polskiego. RYSUNEK wersja polska aplikacji, a następnie wersja angielska po zmianie języka na przykładzie strony ustawień.
