\chapter{Wstęp i cel pracy} 
W ostatnich latach obserwuje się wzrost zainteresowania technologiami wspierającymi samopomoc, zdrowie psychiczne oraz treningi fizjologiczne, w tym trening oddechu. Aplikacje mobilne pełnią w tym zakresie istotną rolę, umożliwiając użytkownikom łatwy dostęp do zindywidualizowanych programów treningowych, monitorowania postępów oraz technik relaksacyjnych.

Trening oddechowy może mieć korzystny wpływ na zdrowie i być pomocny dla szerokiego grona osób. Może on służyć: 
\begin{itemize}
  \item sportowcom - poprawienie wytrzymałości;
  \item pacjentom z zespołami bólowymi i chorobami układu oddechowego takimi jak astma, przewlekła obturacyjna choroba płuc czy wady klatki piersiowej - zwiększenie tolerancji wysiłku, ogólnej wydolności fizycznej oraz maksymalnego poboru tlenu;
  \item osobom z zaburzeniami lękowymi lub doświadczającym przewlekłego stresu - ukojenie lęku i łagodzenie stresu;
  \item osobom zmagającym się z problemami ze snem - poprawa jakości snu;
  \item instrumentalistom dętym - polepszenie zadęcia;
  \item każdej osobie chcącej pracować nad oddechem - polepszenie układu odpornościowego, zwiększenie ilości tlenu w organizmie.
\end{itemize}

Lorem ipsum dolor sit amet, consectetur adipiscing elit. Vivamus elementum arcu nec blandit aliquam. Integer eros dolor, molestie eget dictum quis, luctus sit amet sapien. Proin dignissim felis in ornare volutpat. Morbi vulputate rutrum efficitur. Ut vehicula vehicula metus, et iaculis tortor mattis vel. Nam blandit, arcu quis ultricies blandit, libero ante commodo augue, in accumsan dui leo at orci. Phasellus in augue et velit pulvinar malesuada ut et sem. Nulla vehicula nibh eu odio sollicitudin sagittis. Praesent condimentum semper neque, tincidunt luctus nisl scelerisque sed. Orci varius natoque penatibus et magnis dis parturient montes, nascetur ridiculus mus.
