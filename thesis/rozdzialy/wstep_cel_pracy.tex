\chapter{Wstęp i cel pracy} 

\section{Wprowadzenie (problemy) \Hania}
\todoHania{in progress}
W ostatnich latach obserwuje się wzrost zainteresowania technologiami wspierającymi samopomoc, zdrowie psychiczne oraz treningi fizjologiczne, w tym trening oddechu. Aplikacje mobilne pełnią w tym zakresie istotną rolę, umożliwiając użytkownikom łatwy dostęp do zindywidualizowanych programów treningowych, monitorowania postępów oraz technik relaksacyjnych.

\section{Motywacje (korzyści) \Hania}
\todoHania{in progress}
Wsparcie zdrowia psychicznego i fizycznego poprzez ułatwienie wykonywania ćwiczeń oddechowych. 
Możliwość wykorzystania dostępności aplikacji mobilnych 

Badania naukowe wskazują na terapeutyczny potencjał regularnych ćwiczeń oddechowych. Systematyczne przeglądy literatury potwierdzają, że techniki takie jak oddychanie przez zaciśnięte usta (ang. \textit{pursed lip breathing}), oddychanie przeponowe oraz techniki jogiczne (np. pranajama) przyczyniają się do istotnej poprawy jakości życia oraz zmniejszenia odczuwania duszności u pacjentów z przewlekłymi chorobami układu oddechowego, takimi jak POChP czy astma \cite{Breathing_Review}. Analiza ta, obejmująca również metody takie jak technika Buteyko \cite{Buteyko_Trial} czy metoda Papworth \cite{Papworth_Trial}, wskazuje na wysoki poziom bezpieczeństwa tych interwencji - w przeglądanych badaniach nie odnotowano istotnych działań niepożądanych związanych z samym treningiem. Jest to kluczowy argument przemawiający za zasadnością udostępniania tych narzędzi w formie aplikacji mobilnej do samodzielnego stosowania. Co więcej, poprawa jakości życia związana ze zdrowiem (ang. \textit{Health-Related Quality of Life}) obserwowana u pacjentów stosujących te techniki często przekraczała minimalną różnicę istotną klinicznie, co potwierdza, że korzyści te są odczuwalne i znaczące dla pacjenta w codziennym funkcjonowaniu. Wykazano również bezpośredni wpływ tych ćwiczeń na poprawę jakości snu, co ma szczególne znaczenie w rehabilitacji osób starszych cierpiących na przewlekłe schorzenia płuc \cite{COPD_Sleep}. Mechanizm ten wiązany jest ze stymulacją aktywności nerwu błędnego poprzez powolne, rytmiczne oddychanie, co sprzyja relaksacji i regulacji układu autonomicznego.

W kontekście zdrowia psychicznego, interwencje oparte na świadomej kontroli oddechu zyskują na znaczeniu jako metody wspierające leczenie zaburzeń nastroju. Badania kliniczne dowodzą, że zarówno specyficzne metody, takie jak metoda Wima Hofa (łącząca techniki oddechowe z ekspozycją na zimno), jak i tradycyjne powolne oddychanie, mogą skutecznie redukować objawy depresji, lęku oraz poziom odczuwanego stresu. W badaniu z udziałem kobiet z wysokim poziomem objawów depresyjnych odnotowano redukcję tych objawów o 24\%, lęku o 27\% oraz postrzeganego stresu o 20\% bezpośrednio po interwencji. Co istotne, poprawa ta utrzymywała się również w okresie obserwacji po 3 miesiącach, gdzie blisko połowa badanych zgłaszała łagodne objawy lub ich brak. Warto również odnotować, że niektóre z tych metod mogą być szczególnie skuteczne w zmniejszaniu ruminacji, czyli uporczywego nawracania negatywnych myśli \cite{WimHof_Depression}.

Równocześnie, rozwój technologii mobilnych otwiera nowe możliwości w zakresie dostępności rehabilitacji oddechowej. Metaanalizy wskazują, że aplikacje mobilne wspierające samodzielny trening są skutecznym narzędziem w rehabilitacji pulmonologicznej, oferując efekty porównywalne do tradycyjnych metod. Kluczowym czynnikiem sukcesu jest tu systematyczność - wykazano korelację między częstotliwością korzystania z aplikacji a redukcją objawów choroby (mierzoną np. w skali CAT). Aplikacje umożliwiające personalizację planów treningowych oraz dostosowanie intensywności ćwiczeń do indywidualnych możliwości pacjenta są wskazywane jako najbardziej obiecujący kierunek rozwoju, pozwalający na utrzymanie długoterminowego zaangażowania użytkownika \cite{MobileApp_Rehab}. Aplikacja ReSpire wpisuje się w ten trend, oferując konfigurowalne wzorce oddechowe, które mogą być dostosowane do zaleceń medycznych i preferencji użytkownika.

Może on służyć:
\begin{enumerate}
  \item sportowcom - poprawienie wytrzymałości;
  \item pacjentom z zespołami bólowymi i chorobami układu oddechowego takimi jak astma, przewlekła obturacyjna choroba płuc czy wady klatki piersiowej - zwiększenie tolerancji wysiłku, ogólnej wydolności fizycznej oraz maksymalnego poboru tlenu \cite{Breathing_Review};
  \item osobom z zaburzeniami lękowymi lub doświadczającym przewlekłego stresu - ukojenie lęku i łagodzenie stresu \cite{WimHof_Depression};
  \item osobom zmagającym się z problemami ze snem - poprawa jakości snu \cite{COPD_Sleep};
  \item instrumentalistom dętym - polepszenie zadęcia;
  \item każdej osobie chcącej pracować nad oddechem - polepszenie układu odpornościowego, zwiększenie ilości tlenu w organizmie.
\end{enumerate}

\section{Cel pracy \Hania}
Celem pracy jest opracowanie projektu oraz implementacja wieloplatformowej aplikacji mobilnej służącej do przeprowadzania treningu oddechowego opartego na konfigurowalnych wzorcach oddechowych obejmujących wdech, retencję, wydech i regenerację, z możliwością dostosowania parametrów dźwiękowych oraz językowych (polski i angielski), a także z wbudowaną wizualizacją przebiegu treningu.

\section{Podział prac w zespole \Jakub}



\section{Struktura rozdziałów \Hania}
W celu przejrzystego przedstawienia realizacji projektu i ułatwienia odbioru pracy, została ona podzielona na dziewięć rozdziałów opisanych poniżej.
	
W rozdziale pierwszym przedstawiono wprowadzenie do tematu pracy, przedstawiono problemy i motywacje podjęcia pracy, jej cel oraz podział obowiązków w zespole.
	
Rozdział drugi opisuje aktualny stan wiedzy, obejmujący istniejące rozwiązania aplikacji mobilnych wspierających trening oddechowy oraz aspekty medyczne i sportowe treningu oddechowego. Przedstawiono również opisy podsumowujący ważne elementy i brakujące funkcjonalności w tych rozwiązaniach.
	
W rozdziale trzecim określono główne założenia projektowe, w których przedstawiono grupę docelową, korzyści płynące z projektu oraz zakładane główne funkcjonalności aplikacji. Rozdział ten definiuje również przewidywanych użytkowników oraz podstawowe wymagania wobec systemu.
	
Rozdział czwarty zawiera analizę wymagań, w tym modele przypadków użycia, modele i diagramy klas oraz dźwięków wykorzystywanych w projekcie.

W rozdziale piątym przedstawiono projekt rozwiązania, obejmujący architekturę systemu, logikę aplikacji, interfejs użytkownika oraz strukturę danych.

Rozdział szósty opisuje implementację aplikacji, w tym strukturę plików i modułów oraz wybrane fragmenty kodu, obejmujące najważniejsze elementy projektu. Zwrócono w nim uwagę na zastosowane technologie i narzędzia programistyczne.

W rozdziale siódmym zamieszczono instrukcję użytkownika opisującą sposób korzystania z aplikacji.

Rozdział ósmy poświęcono testowaniu, obejmującemu testy jednostkowe, integracyji modułów, systemowe i użytkowe. Zaprezentowano w nim wyniki testów oraz ocenę poprawności działania aplikacji w różnych scenariuszach użycia.

Dziewiąty rozdział, zawiera podsumowanie pracy, omówienie osiągniętych rezultatów, napotkanych trudności oraz wnioski dotyczące projektu. Wskazano również możliwe kierunki przyszłych usprawnień i rozszerzeń funkcjonalnych systemu.

