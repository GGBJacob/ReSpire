\chapter{Wstęp i cel pracy} 

\section{Wprowadzenie (problemy) \Hania}
W ostatnich latach obserwuje się wzrost zainteresowania technikami wspierającymi zdrowie psychiczne oraz treningi fizjologiczne.Ważną rolę w tym obszarze odgrywają ćwiczenia oddechowe wykorzystywane w sporcie, terapii i codziennym dbaniu o dobre samopoczucie. Aplikacje mobilne pełnią w tym zakresie istotną rolę, sprzyjając ich popularności. Umożliwiają użytkownikom stały dostęp do zindywidualizowanych programów treningowych z dowolnego miejsca i w dowolnym czasie, pozwalając użytkownikowi dopasować przebieg ćwiczenia do własnych preferencji i celów.

Niniejsza praca koncentruje się na zaprojektowaniu aplikacji mobilnej, która umożliwia wykonywanie treningów oddechowych z wykorzystaniem konfigurowalnych wzorców oddechowych i czytelnej prezentacji ich przebiegu.

\section{Motywacje (korzyści) \Hania}
\todoHania{dodać biblo}
Motywacją do realizacji projektu jest stworzenie narzędzia ułatwiającego wykonywanie ćwiczeń oddechowych, które mogą wspierać zarówno zdrowie psychiczne, jak i fizyczne. Regularne treningi oddechowe są stosowane w różnych grupach użytkowników. Może on służyć:
\begin{itemize}
  \item sportowcom wykorzystującym je do poprawy wytrzymałości \cite{sport};
  \item pacjentom z chorobami układu oddechowego, takimi jak astma wykorzystującym je do poprawy stanu zdrowia \cite{asthma};
  \item osobom z zaburzeniami lękowymi lub doświadczającym przewlekłego stresu wykorzystującym je do ukojenia lęku i łagodzenie stresu;
  \item osobom zmagającym się z problemami ze snem wykorzystującym je do poprawy jakości snu \cite{sleep}; 
  \item każdej osobie chcącej pracować nad oddechem, która chce wesprzeć układ odpornościowy i zwiększyć ilość tlenu w organizmie.
\end{itemize}

Mobilny charakter aplikacji sprawia, że trening staje się bardziej dostępny. Użytkownik może wykonywać go samodzielnie, w dowolnym miejscu, w tempie dostosowanym do swoich potrzeb. To uzasadnia potrzebę stworzenia rozwiązania, które umożliwia wygodną konfigurację wzorców oddechowych, odpowiednie prowadzenie treningu oraz przejrzystą prezentację jego przebiegu.

\section{Cel pracy \Hania}
Celem pracy jest opracowanie projektu oraz implementacja wieloplatformowej aplikacji mobilnej służącej do przeprowadzania treningu oddechowego opartego na konfigurowalnych wzorcach oddechowych obejmujących wdech, retencję, wydech i regenerację, z możliwością dostosowania parametrów dźwiękowych oraz językowych (polski i angielski), a także z wbudowaną wizualizacją przebiegu treningu.

\section{Struktura rozdziałów \Hania}
W celu przejrzystego przedstawienia realizacji projektu i ułatwienia odbioru pracy, została ona podzielona na dziewięć rozdziałów opisanych poniżej.
	
W rozdziale pierwszym przedstawiono wprowadzenie do tematu pracy, przedstawiono problemy i motywacje podjęcia pracy, jej cel oraz podział obowiązków w zespole.
	
Rozdział drugi opisuje aktualny stan wiedzy, obejmujący istniejące rozwiązania aplikacji mobilnych wspierających trening oddechowy oraz aspekty medyczne i sportowe treningu oddechowego. Przedstawiono również opisy podsumowujący ważne elementy i brakujące funkcjonalności w tych rozwiązaniach.
	
W rozdziale trzecim określono główne założenia projektowe, w których przedstawiono grupę docelową, korzyści płynące z projektu oraz zakładane główne funkcjonalności aplikacji. Rozdział ten definiuje również przewidywanych użytkowników oraz podstawowe wymagania wobec systemu.
	
Rozdział czwarty zawiera analizę wymagań, w tym modele przypadków użycia, modele i diagramy klas oraz dźwięków wykorzystywanych w projekcie.

W rozdziale piątym przedstawiono projekt rozwiązania, obejmujący architekturę systemu, logikę aplikacji, interfejs użytkownika oraz strukturę danych.

Rozdział szósty opisuje implementację aplikacji, w tym strukturę plików i modułów oraz wybrane fragmenty kodu, obejmujące najważniejsze elementy projektu. Zwrócono w nim uwagę na zastosowane technologie i narzędzia programistyczne.

W rozdziale siódmym zamieszczono instrukcję użytkownika opisującą sposób korzystania z aplikacji.

Rozdział ósmy poświęcono testowaniu, obejmującemu testy jednostkowe, integracyji modułów, systemowe i użytkowe. Zaprezentowano w nim wyniki testów oraz ocenę poprawności działania aplikacji w różnych scenariuszach użycia.

Dziewiąty rozdział, zawiera podsumowanie pracy, omówienie osiągniętych rezultatów, napotkanych trudności oraz wnioski dotyczące projektu. Wskazano również możliwe kierunki przyszłych usprawnień i rozszerzeń funkcjonalnych systemu.

